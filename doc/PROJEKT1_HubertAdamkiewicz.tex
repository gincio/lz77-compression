\documentclass{article}
\usepackage[utf8]{inputenc}
\usepackage{polski}
\selecthyphenation{polish}
\usepackage{graphicx}
\usepackage{hyperref}
\usepackage{enumerate}

\def\code#1{\texttt{#1}}

\title{Kodowanie słownikowe danych o strukturze bajtowej}
\author{Hubert Adamkiewicz}
\date{Styczeń 2021}

\begin{document}
\maketitle

\section{Opis problemu}
Zadanie polega na zrealizowaniu prostego kodera i dekodera wykorzystującego metodę LZ77
(dla ustalenia uwagi proszę stosować metodę w wariancie przedstawionym w materiałach
wykładowych).
Przyjmujemy następujące założenia:
\begin{enumerate}
\item Kodowane pliki mają strukturę bajtową (tzn. mogą zawierać dane z dowolnego
problemu, w którym występuje co najwyżej \textbf{256} różnych „komunikatów”).
\item Długość \textit{bufora słownika} wynosi \texybf{256} bajtów, a \textit{bufora wejściowego (look-ahead buffer)}
\textbf{15} bajtów, tzn. pojedynczy \textit{wskaźnik do słownika} ma rozmiar \textbf{2.5} bajta (\textbf{1} bajt
reprezentujący położenie kopii fragmentu znalezionej w słowniku czyli \textit{offset}, \textbf{0.5} bajta
do zakodowania \textit{długości kopii} oraz \textbf{1} bajt zawierający „komunikat” bezpośrednio po
znalezionym fragmencie).
\item Rozwiązanie ma mieć postać DWÓCH gotowych do użycia funkcji/skryptów/aplikacji,
tzn. \textit{kodera} i \textit{dekodera}.
\item Koder ma dwa argumenty wejściowe, tzn. (1) nazwa wraz z rozszerzeniem
kodowanego pliku oraz (2) nazwa (wraz z dowolnie zdefiniowanym rozszerzeniem, na
przykład \textbf{.xxx}) pliku zakodowanego, który ma być zapisany w tym samym folderze,
co plik kodowany.
\item \textit{Dekoder} ma również dwa argumenty wejściowe, tzn. (1) nazwa (wraz z zastosowanym
rozszerzeniem, na przykład \textbf{.xxx}) dekodowanego pliku oraz (2) nazwa (wraz z
zadanym rozszerzeniem) pliku rozkodowanego, który ma być zapisany w tym samym
folderze, co plik dekodowany.
Zrealizowane rozwiązania powinny być przetestowane na wybranych przykładach (np. plikach
w formacie \textbf{.txt}).

\section{Realizacja}
TODO: Napisać nowy tekst. Dalej już są teksty z poprzedniego projektu
\subsection{Kwantyzacja}
W celu zredukowania liczby kodowanych informacji dokonałem kwantyzacji skalarnej równomiernej. 
Proces ten polegał na obliczeniu histogramu dla \code{L} przedziałów wartości (gdzie \code{L} -- liczba możliwych poziomów po kwantyzacji), wyznaczenie środków tych przedziałów i ustalenie ich jako nowych wartości sygnału skwantowanego.

\subsection{Kodowanie}
Sygnał skwantowany został zakodowany przy pomocy kodu Huffmana. Jest to metoda bardzo skuteczna, jeżeli znane są prawdopodobieństwa wystąpienia poszczególnych komunikatów. W analizowanym przykładzie takie prawdopodobieństwa są wyznaczane tuż po kwantyzacji, a kompresja przeprowadzana jest na znanych danych, więc zastosowanie tej metody powinno być efektywne.

Aby tego dokonać obliczyłem prawdopodobieństwa występowania poszczególnych komunikatów ze zbioru skwantowanych wartości sygnału na podstawie sygnału wejściowego, wyliczyłem kod Huffmana, a następnie przy pomocy uzyskanego kodu przekonwertowałem tablicę wartości skwantowanych na ciąg bitów.

\section{Uruchomienie projektu}
Projekt został zrealizowany w języku programowania Python w wersji 3. 
Niezbędna jest zatem instalacja interpretera języka python dostępnego \href{https://www.python.org/downloads/windows/}{tutaj} oraz menedżera paczek \code{pip}. W tym celu należy pobrać skrypt instalacyjny \href{https://bootstrap.pypa.io/get-pip.py}{stąd}, a następnie uruchomić instalację poleceniem \code{python3 get-pip.py} w katalogu, w którym znajduje się pobrany plik ze skryptem.
Następnie należy zainstalować niezbędne do uruchomienia kodu biblioteki poleceniem: \\
\code{pip install scipy numpy sounddevice matplotlib huffman json}

Aby uruchomić program należy uruchomić linię komend w katalogu zawierającym plik \textit{run.py} oraz użyć komendy \code{python3 run.py}. Uruchomi to kod realizujący najmocniejszą kompresję o parametrze PSNR zbliżonym do 40dB.

Można również uruchomić polecenie z dodatkowymi parametrami umożliwiającymi wybór pliku do kompresji, nazwy pliku binarnego z zakodowanym sygnałem oraz liczby skwantowanych poziomów sygnału. Aby dokonać modyfikacji tych parametrów należy wywołać polecenie w następującym formacie:\\
\code{python3 run.py -i sygnal.wav -o wynik.bin -l poziomy\_kwantowania}.\\
Przykład:
\code{python3 run.py -i icing.wav -o icing.bin -l 200}. Skrypt załaduje plik \textit{icing.wav}, skwantuje sygnał do 200 poziomów, a zakodowany w formie binarnej ciąg znaków zapisze do pliku \textit{icing.bin}

\section{Wyniki}
W celu prezentacji skuteczności metod opisanych w rozdziale \ref{sec:realizacja} zaprezentuję wyniki dwóch przykładowych uruchomień kodu.
\begin{enumerate}
\item Wyniki dla pliku domyślnego - \textit{music2.wav}:
    \begin{enumerate}
    \item \textbf{Maksymalna kompresja} -- 31 poziomów kwantyzacji, PSNR najbliższy 40dB, sygnał jest rozpoznawalny, ale szumy są dość głośne:\\  
    \code{
        PSNR = 40.13444750137048\\
        BR = 3.6456082260288736\\
        Entropia = 3.605834186540232}
    \item \textbf{Dobra jakość} -- 300 poziomów kwantyzacji, jakość zadowalająca, szum zauważalny przy większym poziomie głośności.\\
    \code{
        PSNR = 60.18797300522537\\
        BR = 6.987400030899106\\
        Entropia = 6.955444109513735}
    \end{enumerate}
\item Wyniki dla pliku wejściowego \textit{icing.wav}:
    \begin{enumerate}
    \item \textbf{Maksymalna kompresja}\\  
    \code{
        PSNR = 39.7806597597401\\
        BR = 3.095083163318919\\
        Entropia = 3.0757936495697034
    }

    \item \textbf{Dobra jakość}\\
    \code{
        PSNR = 60.3009764032355\\
        BR = 6.423872204675545\\
        Entropia = 6.395775973820144
    }
    \end{enumerate}
\item Wyniki dla pliku wejściowego \textit{song.wav}:
\begin{enumerate}
\item \textbf{Maksymalna kompresja}\\  
    \code{
        PSNR = 40.04737187674411\\
        BR = 3.7788559999999998\\
        Entropia = 3.74769605187218
    }

    \item \textbf{Dobra jakość}\\
    \code{
        PSNR = 59.92132157073871\\
        BR = 7.121758\\
        Entropia = 7.082421065471701
    }
    \end{enumerate}
\end{enumerate}

Wyniki dla każdego z trzech plików są podobne. Średnia bitowa jest zbliżona do entriopii, a stosunek sygnału do szumu różni się nieznacznie w zależności od wybranego pliku. Jest to spowodowane wykorzystaniem metod niezwiązanych ściśle z określonym plikiem źródłowym, a obliczenie wartości skwantowanych oraz kodu Huffmana jest niezależne dla każdego z nich.

\end{document}
